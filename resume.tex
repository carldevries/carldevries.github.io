\documentclass[overlapped, 11pt]{res}
\usepackage{enumitem}
\usepackage[a4paper, margin=0.5in]{geometry}
\usepackage{booktabs}

\newcommand{\tabitem}{~~\textbullet~~}
\topmargin=-0.5in
\setlength{\textheight}{11in}
\setlist[itemize]{noitemsep, topsep=0.1em, label=\textbullet, leftmargin=0.15in}

\begin{document}

\name{Carl De Vries}
\address
{
    605 East 16th Street Pella, IA 50219 
    \textbar \ (641) 780-9473 
    \textbar \ carldevries@gmail.com 
}

\begin{resume}

    \begin{itemize}
    
        \item[]\section{EDUCATION}
        
            \textbf{Master of Engineering, Aerospace Engineering} \hfill Expected: December 2021 \\
            \textbf{Bachelor of Science, Aerospace Engineering} \hfill May 2020 \\
            \emph{Iowa State University College of Engineering} \textbar
                \ Ames, IA \\
            Graduate GPA: 4.00 \\
            Undergraduate GPA: 3.85 
            \begin{itemize}
                \item[] Senior Overall Academic Achievement Award
            \end{itemize}
                
           
%            \\[0.5em] %Why does this work and nothing else?
%            \textbf{Associate of Arts, Liberal Arts, Pre-engineering w/ Honors} \\
%            \emph{Des Moines Area Community College} \textbar 
%                \ Boone, IA \hfill December 2014\\
%                GPA: 3.76

        \item[]\section{EXPERIENCE}
        
            \textbf{Earth, Moon, and Mars GN\&C Graduate Intern}\\ 
            \emph{The Charles Stark Draper Laboratory, Inc} \textbar 
                \ Houston, TX \hfill June 2020 - August 2020
            \begin{tabular}{l}
                \tabitem Recommended safe lunar landing site selection algorithm inputs to satisfy broad mission objectives \\
                \tabitem Developed a safe lunar landing site selection algorithm for the ispace autonomous lunar lander \\
                \tabitem Implemented a selection algorithm with mission customized objective and weighting capabilities \\
                \tabitem Developed tunable cost functions to capture geographic hazards as a function of distance using cost maps \\
            \end{tabular}
            
            \textbf{Research Intern}\\
            \emph{MIT Lincoln Laboratory} \textbar 
                \ Lexington, MA \hfill May 2019 - August 2019
            \begin{tabular}{l}
                \tabitem Implemented a 3 degree of freedom missile model and integrated it into an aircraft engagement simulation \\
                \tabitem Updated the engagement simulation from a 2-D flat Earth model to a 3-D spherical Earth model \\
                \tabitem Implemented proportional navigation, gravity bias, and lofting schemes for missile guidance \\
           \end{tabular}

%            \vspace{0.25em}
%            \textbf{Undergraduate Researcher}\\
%            \emph{Multi-Agent Unmanned Systems Laboratory} \textbar 
%                \ Ames, IA \hfill November 2018 - May 2019
%            \begin{tabular}{l}
%                \tabitem Implemented a six degree of freedom, non-linear, fixed-wing aircraft model with LQR feedback control \\
%                \tabitem Developed C code to read and publish IMU sensor data using an Arduino and Robotic Operating System \\
%            \end{tabular}
            
            \vspace{0.25em}
            \textbf{Engineering Co-op}\\
            \emph{The Charles Stark Draper Laboratory, Inc.} \textbar 
                \ Cambridge, MA \hfill January 2018 - July 2018
            \begin{tabular}{l}
                \tabitem Implemented guidance computer simulation software in MATLAB and Simulink \\
                \tabitem Verified performance characteristics between system level models and engineering level gyroscope models \\
                \tabitem Integrated a new gyroscope model into a system simulation and generated data for performance verification \\
                \tabitem Automated data analysis and unit tests to verify a gyroscope model's scale factor and bias implementation \\
            \end{tabular}

            \vspace{0.25em}
            \textbf{Software Engineering Co-op}\\
            \emph{Collins Aerospace (formerly Rockwell Collins)} \textbar 
                \ Cedar Rapids, IA \hfill January 2017 - August 2017
            \begin{tabular}{l}
                \tabitem Verified functional and DO-178B Level A compliance for 75 upgraded Simulink models (2007a - 2016b) \\
                \tabitem Developed graphical and functional flight display software using Simulink and Simulink Coder \\
                \tabitem Decreased build times via script enhancements which omitted unchanged models from the build process \\
            \end{tabular}
            
%            \section{SKILLS}
%                    MATLAB, Simulink, Python, C/C++, Linux, Git, SVN
                
         \section{PROJECTS}
                
            \textbf{Guidance and Navigation of Aerospace Vehicles} \\
                \begin{tabular}{l}
                    \tabitem 3DoF Mars entry simulation using two-phase Zero-Effort-Miss/Zero-Effort-Velocity (ZEM/ZEV) guidance\\
                    \tabitem Ballistic missile intercept simulation using true, pulsed proportional navigation and ZEM guidance\\
                    \tabitem Orbital rendezvous using Clohessy-Wiltshire equations, linearized perturbed guidance, and ZEM/ZEV\\
                    \tabitem Strapdown inertial navigation IMU simulation verified using 3DOF Mars entry flight dynamics\\
                \end{tabular}

            \textbf{Random Signals Analysis and Kalman Filtering}\\
                \begin{tabular}{l}
                    \tabitem Kalman Filter estimation of four aircraft longitudinal states from two measurements using elevator input\\
                    \tabitem Extended Kalman Filter estimation of two-dimensional motion model\\
                \end{tabular}
                
%            \textbf{Digital Feedback Control Systems} \\
%                \begin{tabular}{l}
%                    \tabitem \\
%                \end{tabular}
                
            \textbf{Automatic Control of Flight Vehicles}\\
                \begin{tabular}{l}
                    \tabitem PID and LQR controller design for longitudinal and lateral modes of a Cessna T-37 (Simulink)\\
                \end{tabular}
                
            \textbf{Advanced Engineering Dynamics}\\
                \begin{tabular}{l}
                    \tabitem Space-based solar power satellite in a gravity gradient with quaternions and modified Rodrigues parameters\\
                    \tabitem Nonlinear Lyapunov feedback controller developed for satellite pointing and attitude disturbance rejection\\
                    \tabitem Developed EoM using D'Alembert's principle, Lagrange's equation, and Hamilton's principle\\
                \end{tabular}

%            \textbf{Space Trajectory Optimization} \\
%                \begin{tabular}{l}
%                    \tabitem \\
%                \end{tabular}
                
%            \textbf{Orbital Mechanics} \\
%                \begin{tabular}{l}
%                    \tabitem \\
%                \end{tabular}
                
            \textbf{Astrodynamics II} \\
                \begin{tabular}{l}
                    \tabitem Orbital insertion simulation and trajectory design for a two stage, solid-fuel gravity turn rocket\\
                    \tabitem Lunar free return trajectory simulation using circular, restricted three body (CR3BP) dynamics\\
                \end{tabular}
            
            \textbf{Spacecraft Dynamics and Control} \\
                \begin{tabular}{l}
                    \tabitem Three-axis quaternion feedback CMG controller for satellite multi-target rest-to-rest maneuvers\\
                \end{tabular}
                
%            \textbf{Remote Sensing Technologies} \\
%                \begin{tabular}{l}
%                    \tabitem \\
%                \end{tabular}
                
%            \textbf{Multi-Disciplinary Optimization} \\
%                \begin{tabular}{l}
%                    \tabitem \\
%                \end{tabular}
                
%            \textbf{Computational Methods for Fluid Mechanics and Heat Transfer} \\
%                \begin{tabular}{l}
%                    \tabitem \\
%                \end{tabular}
                
            \textbf{Additional Projects} \\
                \begin{tabular}{l}
                    %\tabitem Developed closed loop discrete-time systems using sampling and reconstruction theory\\
                    %\tabitem Analyzed the response and stability of discrete-time, closed loop transfer functions and state space models\\
                    %\tabitem Simulated Black Brant V \& IX sounding rocket models to analyze single \& multi-stage performance (C++)\\
                    \tabitem Black Brant sounding rocket simulations to analyze single and multi-stage motor performance (C++)\\
                    \tabitem Runge-Kutta-Fehlberg (RK45) adaptive step-size integrator verified using CR3BP solution (C++)\\
                    \tabitem Developed an autonomous vehicle to navigate around a 12 foot track 5 times in 60 seconds (C++)\\
                \end{tabular}
                
             \textbf{Current Coursework} \\
                \begin{tabular}{l}
                    \tabitem \textbf{Orbital Mechanics} \\
                    \tabitem \textbf{Remote Sensing Technologies} \\
                    \tabitem \textbf{Multi-Disciplinary Optimization} \\
                    \tabitem \textbf{Computational Methods for Fluid Mechanics and Heat Transfer} \\
                \end{tabular}

        \item[]\section{ORGANIZATIONS}
            ISU Spaceflight Operations Workshop
                \hfill August 2018 \\
            Vermeer International Leadership Program
                \hfill August 2018 - May 2019 \\
            NASA National Community College Aerospace Scholars
                \hfill September 2013 - February 2014
        
        \end{itemize}
    \end{resume}
\end{document}